\chapter{Spatially Variable Thresholding}
\label{thresholdchapter}


Previous studies, e.g. \cite{JOT_2005}, demonstrated that in SCALES, the SGS dissipation  is proportional to  $\epsilon ^2$;
therefore, one can enhance SCALES by exploiting spatially-varying $\epsilon$  based on local SGS dissipation
$\Pi = -\tau_{ij}^* %\bareps[S]_{ij}
               {{\left.\red\overline{\bl S}_{\bl ij} \rule{-8pt}{6pt}\right.}^{\red {\scriptscriptstyle >} \epsilon}}
               $.
This implies that rate of local-transfer of energy from
energetic-resolved-eddies to unresolved-less-energetic structures can be controlled by varying the thresholding-factor.
Therefore, the idea is to locally vary $\epsilon$ wherever $\Pi$ deviates from a priori defined goal-value.
A decrease of the thresholding level results in the local grid refinement with subsequent rise of the resolved viscous dissipation, while an increase of $\epsilon$ coarsens the mesh resulting in  the growth of the local SGS dissipation.
%
%
%This corresponds to the fact that due to wavelet-thresholding-filter,
%the resolved-scales depend on the thresholding parameter as follows:
%\[
%  \begin{array}{rl}
%     \epsilon \downarrow &\Rightarrow \mbox{build-up scales} , \\
%     \epsilon \uparrow   &\Rightarrow \rm{lose \; some \; scales}.
%  \end{array}
%\]
%\noindent
%
However, in order to vary  $\epsilon$ in a physically consistent fashion, it should follow the local flow structures as they evolve in space and time. This necessitates the Lagrangian representation of  $\epsilon$, which is achieved using the  Lagrangian Path-Line Diffusive Averaging approach \cite{JOT_2008}:
\begin{equation}
        \label{eq:LagPathLineDiffAvg1} \vspace*{3pt}
        \partial_{t}{\epsilon}
        + \bareps[u]_j \partial_{x_j}{\epsilon}
        = - {\rm forcing_{term} }
        + \nu_{\epsilon} \partial_{x_jx_j}^2{\epsilon} .
\end{equation}
%
For the sake of efficiency, instead of solving Eq.~(\ref{eq:LagPathLineDiffAvg1}) for the evolution of $\epsilon$, the linear-interpolation along characteristics, similar to the idea of Meneveau et. al \cite{Meneveau_LagrangianSGS_JFM_1996}, is performed
\begin{equation}
        \label{eq:LagPathLineDiffAvg2} \vspace*{3pt}
        \frac 1 {\Delta t} \left[ \epsilon^{\rm new} \left( {\mathbf x},t+\Delta t\right)
                                - \epsilon^{\rm old} \left( {\mathbf x}-\bareps[{\mathbf u}] \Delta t,t\right) \right]
        = {\rm - forcing_{term} }.
\end{equation}
The use of linear interpolation results in sufficient numerical diffusion, thus, eliminating the need for explicit diffusion.
%%%%%%%%%%%%%%%%%%%%%%%%%%%%%%%%%%%%%%%%%%%%%%%%%%%%%%%%%%%%%%%%%% Possible For Deletion
%
%Results for a scalar-transport test case showed a very diffusive transport as expected.
%The forcing term is used to adjusts the threshold value using a simple feedback procedure that is based on   prescribed goal-value of SGS dissipation.
%%%%%%%%%%%%%%%%%%%%%%%%%%%%%%%%%%%%%%%%%%%%%%%%%%%%%%%%%%%%%%%%%%
%
%\noindent
%To visually (quantitatively) ensure the that linear-interpolation along characteristics is diffusive enough, a 2D/3D scalar-transport case
% ($
%        \label{eq:2D3DtestIterpolation} \vspace*{3pt}
%        \partial_{t}{U}
%        + \left( C \cdot \nabla \right) U
%        = 0
%$)
%where epsilon was initialized inside an arbitrarily (in 2D triangular and circular) spot and then transported along characteristic lines with gird adapted to $\epsilon$ itself was preformed. The results showed a very diffusive scalar-transport as expected.
%\noindent
The proposed spatially variable thresholding strategy  ensures that the wavelet threshold is not {\em a priori} prescribed but determined on the fly by desired turbulence resolution.
%
%
%\subsecspace
%\subsection{Forcing Term}
%
%
In this work two different mechanisms for the forcing term are studied:
\begin{description}
%
  \item[FT1]{   The local fraction SGSD (FSGSD) is defined as
               $\frac{\Pi }  {  \varepsilon_{\rm res}  +  \Pi }$,
               where \linebreak[4]
               \mbox{$\varepsilon_{\rm res} = 2 \nu
               %\bareps[S]_{ij} \bareps[S]_{ij}
               {{\left.\red\overline{\bl S}_{\bl ij} \rule{-8pt}{6pt}\right.}^{\red {\scriptscriptstyle >} \epsilon}}
               {{\left.\red\overline{\bl S}_{\bl ij} \rule{-8pt}{6pt}\right.}^{\red {\scriptscriptstyle >} \epsilon}}$}
               is the resolved viscous dissipation.
               The idea is to maintain FSGD at a ``Goal'' value which means retain $\Pi$ at
               $\varepsilon_{\rm res} \frac {\rm Goal}{1-{\rm Goal}}$.
               The first forcing type (FT1) is an attempt to implement this while normalizing the forcing term based on its time average:
%
              \begin{equation}
                  \label{eq:FT1_3} \vspace*{3pt}
                  {\rm  forcing_{term} } =
                  \epsilon^{\rm old} \left( {\mathbf x}-\bareps[{\mathbf u}] \Delta t,t\right)
                  C_{{\rm f}_{\epsilon}} \;
                  \frac {\Pi - \varepsilon_{\rm res}  \frac {\rm Goal}{1-{\rm Goal}}} {\rm TAF} ,
              \end{equation}
%
             \noindent
             where ${\rm TAF}$ stands for the time average of the forcing, is the time average of
             $|\Pi - \varepsilon_{\rm res}  \frac {\rm Goal}{1-{\rm Goal}}|$.
             %and $C_{f_{\epsilon}}$.
             %is a constant coefficient which is set to 400.
             %The intentional choice of $400$ is to make the time response of FT1 about three to four times faster than
             %large eddy turnover time which is discussed in the next section.
             The forcing constant coefficient, $C_{{\rm f}_{\epsilon}}$, is intentionally set to 400
             in order to make the time response of FT1 about three to four times faster than
             large eddy turnover time which is discussed in the next section.
}
 \vspace{6pt}
  \item[FT2]{   In this approach, the variations of local-FSGSD is controlled directly based on the goal-value
               in conjunction with a relaxation time parameter (time-scale), $\tau_{\epsilon}$,
%
               \begin{equation}
                   \label{eq:FT2_1} \vspace*{3pt}
                   {\rm  forcing_{term} } =
                   \epsilon^{\rm old} \left( {\mathbf x}-\bareps[{\mathbf u}] \Delta t,t\right)
                   \frac 1 \tau_{\epsilon}
                   \left( \frac{\Pi }  {  \varepsilon_{\rm res}  +  \Pi }  -  {\rm Goal} \right).
               \end{equation}
%
               Following the time-varying threshold studies \cite{POF_2010},
               a time-scale associated to the characteristic rate-of-strain is chosen:
               $\tau_{\epsilon}^{-1} = \langle {|
               %\bareps[S]
               {{\left.\red\overline{\bl S}_{\bl ij} \rule{-8pt}{6pt}\right.}^{\red {\scriptscriptstyle >} \epsilon}}
              %{{\left.\red\overline{\bl S}_{\bl ij} \rule{-8pt}{10pt}\right.}^{\red {\scriptscriptstyle >} \epsilon}}
               |} \rangle$.

}
\end{description}


%\begin{itemize}
%  \item sss
%  \item sss
%\end{itemize}
%
%\begin{enumerate}
%  \item ssss
%  \item ssss
%\end{enumerate}






