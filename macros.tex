\newcommand{\bt}[1]{\textbf{#1}} 					% bold text
\newcommand{\ds}{\displaystyle} 					% displaystyle
\newcommand{\rbr}[1]{\left(#1\right)}				% round brackets
\newcommand{\rbrl}[1]{\left(#1\right.}				% round brackets (left)
\newcommand{\rbrr}[1]{\left.#1\right)}				% round brackets (right)
\newcommand{\sbr}[1]{\left[#1\right]}				% square brackets
\newcommand{\fbr}[1]{\left\{#1\right\}}				% curly bracket
\newcommand{\vbr}[1]{\left.#1\right|}				% right line bracket 
\newcommand{\vbl}[1]{\left|#1\right.}				% left line bracket
\newcommand{\fbrl}[1]{\left\{#1\right.}				% left curly bracket
\newcommand{\fbrr}[1]{\left.#1\right\}}				% right curly bracket
\newcommand{\dbr}[1]{\left\|#1\right\|}				% double line brackets
\newcommand{\lbr}[1]{\left|#1\right|}				% line brackets
\newcommand{\md}[1]{\textmd{#1}}					% normal text
\renewcommand{\it}[1]{\textit{#1}}					% italic text
\newcommand{\bh}[1]{\hat{\bt{#1}}}					% unit vectors = bold + hat
\newcommand{\pt}{\partial}							% \partial
\renewcommand{\div}{\md{div\,}}						% div
\newcommand{\gr}{\md{grad\,}}						% grad
\newcommand{\cl}{\md{curl\,}}						% curl
\newcommand{\vr}{\varepsilon}						% \varepsilon 
\newcommand{\ges}{\epsilon}							% \epsilon
\newcommand{\bs}{$\blacksquare$}					% black square (QED)
\newcommand{\la}{\ \Longrightarrow\ }				% \Longrightarrow + space
\newcommand{\vp}{\varphi}							% \varphi
\newcommand{\olr}[1]{\overleftrightarrow{#1}}		% jargon tensorial notation
\newcommand{\mb}[1]{\boldsymbol{#1}}				% bold symbol in math environment
\newcommand{\ga}{\alpha}							% \alpha
\newcommand{\gb}{\beta}								% \beta
\newcommand{\gG}{\gamma}							% \gamma
\newcommand{\gd}{\delta}							% \delta
\newcommand{\gD}{\Delta}							% \Delta
\newcommand{\gs}{\sigma}							% \sigma
\newcommand{\gl}{\lambda}							% \lambda
\newcommand{\gL}{\Lambda}							% \Lambda
\newcommand{\go}{\omega}							% \omega
\newcommand{\gth}{\theta}							% \theta
\newcommand{\mg}[1]{\mathbb{#1}}					% Bold letters in math mode
\newcommand{\spn}{\md{span}}						% span in normal text
\newcommand{\ra}{\ \rightarrow\ }					% \rightarrow + space
\newcommand{\rl}{\leftrightarr}						% \leftrightarr
\newcommand{\tc}{\checkmark}						% \checkmark
\newcommand{\sgn}{\md{sign}}						% sign
\newcommand{\tcl}{\textcolor}						% \textcolor
\newcommand{\abr}[1]{\left \langle#1\right \rangle}	% angle brackets
\newcommand{\txt}[1]{\texttt{#1}}					% \texttt
\newcommand{\ol}[1]{\overline{#1}}					% overline 
\renewcommand{\exp}[1]{\md{exp} \rbr{#1}}			% exp
\renewcommand{\ln}[1]{\md{ln} \rbr{#1}}				% ln
\newcommand{\mc}{\mathcal}							% calligraphic
\newcommand{\uc}[1]{$^{\md{\underline{#1}}}$}			% underline
\newcommand{\ssc}[1]{\textsuperscript{#1}}			% superscript
\newcommand{\rtm}{\ssc{\textregistered}}			% registered tm sign
\newcommand{\tm}{\ssc{\texttrademark}}				% tm sign
\newcommand\chap[1]{%
  \chapter*{#1}%
  \addcontentsline{toc}{chapter}{#1}}
\newcommand\sect[1]{%
  \section*{#1}%
  \addcontentsline{toc}{section}{#1}}
\newcommand{\tb}{\textbullet}

\def\etal{{\em et al.\ }}
\def\etc{etc.\ }
\def\eg{{\em e.g.\ }}
\def\ie{{\em i.e.\ }}
\def\etaT{\eta_{\mbox{\tiny$T$}}}
\def\alphaT{\alpha_{\mbox{\tiny$T$}}}
\def\drawline#1#2{\raise 2.5pt\vbox{\hrule width #1pt height #2pt}}
\def\spacce#1{\hskip #1pt}
\def\solid{\drawline{24}{.5}\nobreak\ }
\def\bdash{\hbox{\drawline{4}{.5}\spacce{2}}}
\def\dashed{\bdash\bdash\bdash\bdash\nobreak\ }
